\section{Introduction\label{introduction}}
\emph{Scatterplot3d} is an \RR\ package for the visualization of
multivariate data in a three dimensional space.
\RR\ itself is ``A Language and Environment for Statistical Computing'' (\cite{r-ref}) and a freely
available statistical software package implementing that language, see
\url{http://www.R-project.org/}.

Basically \sdd\ generates a scatter plot in the 3D space using a parallel
projection.  Higher dimensions (fourth, fifth, etc.) of the data can be
visualized to some extent using, e.g. different colors, symbol types or
symbol sizes.

The following properties of \sdd\ will be further described and discussed
in the present paper:
%
A plot is generated entirely by using interpreted \RR\ graphics functions,
so the appearance of the plot is consistent with other \RR\ graphics.
Such a behavior is % extremely
important for publications.
Most features of the \RR\ graphics system can be applied in \sdd , among
them are vectorizing of colors or plotting symbols and mathematical
annotation (\cite{murrell00}).
The latter means whole formulas with e.g.\ greek letters and mathematical
symbols inside can be added into plots using a \LaTeX\ like syntax.
%
\emph{Scatterplot3d} can be easily extended e.g., by adding additional
points or drawing regression lines or planes into an already generated
plot (via function closures, see below).
The package is platform independent and can easily be installed,
because it only requires an installed version of \RR.

This paper is structured as follows:
%
In Section \ref{design} the design of \sdd\ will be described, followed by
remarks on the extensibility of the function in Section \ref{extend}.
%
Some examples (including code and results) on generated and real world data
are provided in Section \ref{examples}.
%
We present other \RR\ related 3D ``tools'' in Section \ref{tools}, followed
by the conclusion in Section \ref{conclusion}.
%
In the Appendix the source code as well as the help page of \sdd\ are
printed.

\RR\ and \sdd\ are available from \emph{CRAN} (Common \RR\ Archive
Network), i.e. \url{http://CRAN.R-Project.org } or one of its mirrors.

