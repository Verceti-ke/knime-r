\HeaderA{pmml.arules}{Generate PMML for arules objects}{pmml.arules}
\aliasA{pmml.itemsets}{pmml.arules}{pmml.itemsets}
\aliasA{pmml.rules}{pmml.arules}{pmml.rules}
\keyword{interface}{pmml.arules}
\begin{Description}\relax
Generate the PMML (Predictive Model Markup Language) representation of
a rules or an itemset object from package \pkg{arules}. 
The PMML can then be imported into other systems that accept PMML.
\end{Description}
\begin{Usage}
\begin{verbatim}
## S3 method for class 'rules':
pmml(model, model.name="arules_Model", 
    app.name="Rattle/PMML",
    description="arules association rules model", copyright=NULL, ...)
## S3 method for class 'itemsets':
pmml(model, model.name="arules_Model", 
    app.name="Rattle/PMML",
    description="arules frequent itemsets model", copyright=NULL, ...)
\end{verbatim}
\end{Usage}
\begin{Arguments}
\begin{ldescription}
\item[\code{model}] an rules or itemsets object.
\item[\code{model.name}] a name to give to the model in the PMML.
\item[\code{app.name}] the name of the application that generated the PMML.
\item[\code{description}] a descriptive text for the header of the PMML.
\item[\code{copyright}] the copyright notice for the model.
\item[\code{...}] further arguments passed to or from other methods.
\end{ldescription}
\end{Arguments}
\begin{Details}\relax
The generated PMML can be imported into any PMML consuming
application.
\end{Details}
\begin{Author}\relax
Michael Hahsler (\email{michael@hahsler.net})
\end{Author}
\begin{References}\relax
Package \pkg{arules} home page: 
\url{http://r-forge.r-project.org/projects/arules}

Package home page: \url{http://rattle.togaware.com}

PMML home page: \url{http://www.dmg.org}
\end{References}
\begin{SeeAlso}\relax
\code{\LinkA{pmml}{pmml}}.
\end{SeeAlso}

