\HeaderA{pmml.rpart}{Generate PMML for an rpart object}{pmml.rpart}
\keyword{interface}{pmml.rpart}
\keyword{tree}{pmml.rpart}
\begin{Description}\relax
Generate the PMML (Predictive Model Markup Language) representation of
an \pkg{rpart} object (classification tree).  The rpart object
(currently expected to be a classification tree) is converted into a
PMML representation. The PMML can then be imported into other systems
that accept PMML.
\end{Description}
\begin{Usage}
\begin{verbatim}
## S3 method for class 'rpart':
pmml(model, model.name="RPart_Model", app.name="Rattle/PMML",
     description="RPart decision tree model", copyright=NULL, ...)
\end{verbatim}
\end{Usage}
\begin{Arguments}
\begin{ldescription}
\item[\code{model}] an rpart object.
\item[\code{model.name}] a name to give to the model in the PMML.
\item[\code{app.name}] the name of the application that generated the PMML.
\item[\code{description}] a descriptive text for the header of the PMML.
\item[\code{copyright}] the copyright notice for the model.
\item[\code{...}] further arguments passed to or from other methods.
\end{ldescription}
\end{Arguments}
\begin{Details}\relax
The generated PMML can be imported into any PMML consuming
application, such as Teradata Warehouse Miner and DB2. Generally,
these applications convert the PMML into SQL for execution across a
database.

Teradata, for example, generates a single SELECT statement to
implement a decision tree. In the Examples section below, we use the
rpart example to build a model stored in the variable fit. A segment
of the PMML for this model is:

\begin{alltt}
  <Node score="absent" recordCount="81">
   <True/>
   <Node score="absent" recordCount="62">
    <SimplePredicate field="Start" operator="greaterOrEqual"
                     value="8.5"/>
    <Node score="absent" recordCount="29">
     <SimplePredicate field="Start" operator="greaterOrEqual"
                      value="14.5"/>
    </Node>
    <Node score="absent" recordCount="33">
     <SimplePredicate field="Start" operator="lessThan"
                      value="14.5"/>
     <Node score="absent" recordCount="12">
      <SimplePredicate field="Age" operator="lessThan"
                       value="55"/>
     </Node>
     <Node score="absent" recordCount="21">
      <SimplePredicate field="Age" operator="greaterOrEqual" 
                      value="55"/>
      <Node score="absent" recordCount="14">
       <SimplePredicate field="Age" operator="greaterOrEqual"
                        value="111"/>
      </Node>
      <Node score="present" recordCount="7">
       <SimplePredicate field="Age" operator="lessThan"
                        value="111"/>
      </Node>
     </Node>
    </Node>
   </Node>
   <Node score="present" recordCount="19">
    <SimplePredicate field="Start" operator="lessThan"
                     value="8.5"/>
   </Node>
  </Node>
\end{alltt}

The resulting SQL from Teradata includes:

\begin{alltt}
  CREATE TABLE "MyScores" AS (
    SELECT "UserID",
      (CASE WHEN \_node = 0 THEN 'absent'
            WHEN \_node = 1 THEN 'absent'
            WHEN \_node = 2 THEN 'absent'
            WHEN \_node = 3 THEN 'present'
            WHEN \_node = 4 THEN 'present'
            ELSE NULL END)
            (VARCHAR(8)) AS "Kyphosis"
    FROM
      (SELECT "UserID",
        (CASE WHEN ("Start" >= 8.5) AND ("Start" >= 14.5)
              THEN 0
              WHEN ("Start" >= 8.5) AND ("Start" < 14.5)
              AND ("Age" < 55)
              THEN 1
              WHEN ("Start" >= 8.5) AND ("Start" < 14.5)
              AND ("Age" >= 55) AND ("Age" >= 111)
              THEN 2
              WHEN ("Start" >= 8.5) AND ("Start" < 14.5)
              AND ("Age" >= 55) AND ("Age" < 111)
              THEN 3
              WHEN ("Start" < 8.5)
              THEN 4
              ELSE -1 END) AS \_node
        FROM "MyData" WHERE \_node IS NOT NULL) A
        WHERE "Kyphosis" IS NOT NULL)
    WITH DATA UNIQUE PRIMARY INDEX ("UserID");
\end{alltt}
\end{Details}
\begin{Author}\relax
\email{Graham.Williams@togaware.com}
\end{Author}
\begin{References}\relax
Package home page: \url{http://rattle.togaware.com}

PMML home page: \url{http://www.dmg.org}

Zementis' useful PMML convert: \url{http://www.zementis.com/pmml_converters.htm}
\end{References}
\begin{SeeAlso}\relax
\code{\LinkA{pmml}{pmml}}.
\end{SeeAlso}
\begin{Examples}
\begin{ExampleCode}
library(rpart)
fit <- rpart(Kyphosis ~ Age + Number + Start, data=kyphosis)
pmml(fit)
\end{ExampleCode}
\end{Examples}

