\HeaderA{pmml}{Generate PMML for an R object}{pmml}
\keyword{interface}{pmml}
\begin{Description}\relax
'pmml' is a generic function implementing S3 methods used to produce
the PMML (Predictive Model Markup Language) representation of an R
model.  The PMML can then be imported into other systems that accept
PMML.
\end{Description}
\begin{Usage}
\begin{verbatim}
pmml(model, model.name="Rattle_Model", app.name="Rattle/PMML",
     description=NULL, copyright=NULL, ...)
\end{verbatim}
\end{Usage}
\begin{Arguments}
\begin{ldescription}
\item[\code{model}] an object to be converted to PMML.
\item[\code{model.name}] a name to give to the model in the PMML.
\item[\code{app.name}] the name of the application that generated the PMML.
\item[\code{description}] a descriptive text for the header of the PMML.
\item[\code{copyright}] the copyright notice for the model.
\item[\code{...}] further arguments passed to or from other methods.
\end{ldescription}
\end{Arguments}
\begin{Details}\relax
The generated PMML can be imported into any PMML consuming
application, such as Teradata Warehouse Miner and DB2. Generally,
these applications convert the PMML into SQL for execution across a
database.
\end{Details}
\begin{Author}\relax
\email{Graham.Williams@togaware.com}
\end{Author}
\begin{References}\relax
Package home page: \url{http://rattle.togaware.com}

PMML home page: \url{http://www.dmg.org}
\end{References}
\begin{SeeAlso}\relax
\code{\LinkA{pmml.kmeans}{pmml.kmeans}},
\code{\LinkA{pmml.rpart}{pmml.rpart}},
\code{\LinkA{pmml.rsf}{pmml.rsf}}.
\end{SeeAlso}

